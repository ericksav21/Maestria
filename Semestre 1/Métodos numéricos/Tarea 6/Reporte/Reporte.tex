\documentclass[12pt]{article}
\usepackage{pgf, tikz}
\usepackage{amsmath, amsfonts, amssymb, graphicx}
\usepackage{subfig}
\usepackage{float}
\usepackage[utf8]{inputenc}
\usepackage[spanish]{babel}
\usepackage{amsthm}

\setlength{\textheight}{23cm} \setlength{\evensidemargin}{0cm}
\setlength{\oddsidemargin}{-.5cm} \setlength{\topmargin}{-3cm}
\setlength{\textwidth}{17.5cm} \setlength{\parskip}{.2cm}


%opening

\begin{document}
	\begin{picture}(80, 80)
	\put(170,0){\hbox{\includegraphics[scale=0.6]{cimat_logo.png}}}
	\end{picture}
	
	\begin{center}
		\begin{huge}
			Centro de Investigación en Matemáticas, A.C.
		\end{huge}
	\end{center}

	\begin{center}
		\begin{large}
			Descripción tarea - Métodos numéricos
		\end{large}
	\end{center}
	
	\begin{center}
		\textbf{Erick Salvador Alvarez Valencia}
	\end{center}

	\begin{center}
		5 de Octubre de 2017
	\end{center}



%\maketitle

%\tableofcontents

\section{Introducción}

\section{Método de potencia inversa}

\subsection{Descripción}
El primer método a describir es el de la potencia inversa, el cuál nos ayuda a encontrar el eigenvalor más pequeño, aunque con una pequeña modificación puede encontrar algún valor que sea muy cercano a un $\delta$ dado.\\
Sea $A \in R^{nxn}$ una matriz no necesariamente simétrica, estamos interesados en encontrar un eigenpar $(\phi, \lambda)$, aplicando el método de la potencia inversa encontraremos un eigenpar tal que el valor de $\lambda$ sea el más pequeño que existe. Ahora definimos a $\delta$ como un valor que está dentro del rango $[-d, d]$ donde $d = ||A||_{\infty}$. Podemos ver entonces que si sumamos $\delta$ a la matriz $A$ obtenemos:\\
$$(A + \delta I)\phi = A\phi + \delta \phi = \lambda \phi + \delta \phi = (\lambda + \delta)\phi$$

De lo anterior se obtiene un nuevo eigenvalor asociado al mismo eigenvector. Por lo que podemos concluir que eligiendo valores de $\delta$ apropiados, podremos encontrar posiblemente todos los eigenpares de la matriz mediante el método de la potencia.\\
En el código proporcionado se realiza una discretización uniforme dentro del rango $[-d, d]$ con un tamaño de paso $\Delta d$, dicha constante la obtenemos como $\Delta d = 2d / N$, el usuario brindará el valor de $N$ al programa. Ahora por cada valor $\delta_i$ y con una cierta tolerancia $\tau$ llamamos a la función de la potencia inversa, la cual nos devolverá un eigenvalor $\mu_i$ y su eigenvector $\phi_i$ asociado, si se cumple que $|\mu_0 - \mu_i| > 0.0001$ entonces podemos considerar a $\mu_i$ y a su respectivo $\phi_i$ como un eigenpar válido. Para asegurarnos de tener una muy buena probabilidad de encontrar todos los eigenpares de una matriz, ejecutamos el método de la potencia con un máximo de 1000 iteraciones.

\subsection{Ejemplo de ejecución}
A continuación se mostrará el resultado de la ejecución del método de la potencia con una matriz de 5x5.\\

\begin{figure}[H]
	\centering
	\subfloat[][Figura 1. Resultado de la ejecución del programa con 50 particiones.]{
		\includegraphics[scale=0.3]{E1.png}
	}\hfill
\end{figure}

En la Figura 1. podemos observar que el programa se ejecutó con una matriz de tamaño 5x5 y con 50 particiones. Lo primero que podemos observar es que el algoritmo nos arrojó todos los eigenpares existentes y el error obtenido en los mismos, dicho error es muy pequeño por lo que podemos concluir que la calidad de los resultados es buena.

\subsection{Compilación y ejecución}
\textbf{Para compilar:} En la carpeta encontraremos los archivos $.c$ y $.h$ con los que se podrá compilar el ejecutable. De la misma forma, en conjunto con los archivos anteriores, también podremos encontrar un Makefile para, en caso de encontrarse en linux, compilar de manera sencilla.

\begin{enumerate}
	\item \textbf{Compilar usando Makefile:} En la terminal, nos colocamos en el directorio donde se encuentre el programa, y ejecutamos el comando $make$, automáticamente se realizará la compilación y se generará el ejecutable. El Makefile también contiene el comando $make clean$ el cual limpiará los archivos generados por la compilación, incluyendo el ejecutable.
	\item \textbf{Compilar directamente:} De la misma forma, podemos compilar directamente usando los siguientes comandos (en terminal):
	\begin{itemize}
		\item gcc -c main.c
		\item gcc -c memo.c
		\item gcc -c funciones\_matriz.c
		\item gcc -c reader.c
		\item gcc -o main main.o memo.o funciones\_matriz.o reader.o -lm
	\end{itemize}
\end{enumerate}

\textbf{Para ejecutar:} Únicamente debemos de usar el comando $./main$ para ejecutar el programa en consola.\\
El programa recibe dos argumentos: Un string que representa el nombre del archivo binario (incluida la extensión) donde se encuentra la matriz; Un entero que representa el número de particiones que se realizará en la discretización. A continuación se mostrará un ejemplo de cómo ejecutar el programa:\\

$./main\ mat005.bin\ 50$

\section{Método de Jacobi para eigenvalores}

\subsection{Descripción}
Este método también nos va a servir para encontrar todos los eigenpares de una matriz $A \in R^{nxn}$ la cual si debe cumplir la restricción de ser simétrica. La idea de este método parte de la siguiente igualdad $A\Phi = \Phi\Lambda$ donde $\Phi$ es una matriz que contiene todos los eigenvectores de $A$ y $\Lambda$ es una matriz diagonal que contiene los eigenvalores asociados a su propio eigenvector.

\end{document}
